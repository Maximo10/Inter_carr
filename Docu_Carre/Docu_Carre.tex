\documentclass[12pt]{article}
\usepackage[spanish]{babel}
\usepackage[utf8]{inputenc}
\usepackage{graphicx}
\usepackage{xcolor}
\usepackage{geometry}
\usepackage{amsmath}
\usepackage[section]{placeins}
\usepackage{hyperref}
\hypersetup{
    colorlinks = true,
    linkcolor = blue,
    urlcolor = cyan
}

\geometry{a4paper, margin=2.5cm}

\title{Documentación Práctica Hilos Coches - Interfaz}
\author{Birhan Fdez Fdez}


\begin{document}
\maketitle
\newpage

\tableofcontents
\newpage

\section{Introducción}

Implementación de una interfaz gráfica para la simulación de carreras de coches usando \textbf{Threads} en Java.  
Se ha añadido sincronización con \texttt{synchronized} para garantizar la correcta actualización del podio.

\section{Estructura del Proyecto}

La aplicación sigue el patrón MVC:

\begin{itemize}
    \item \textbf{Modelo:} \texttt{Coches} y \texttt{Grand\_Prix} contienen la lógica de la carrera, los hilos y el podio.
    \item \textbf{Vista:} \texttt{hello-view.fxml} define los carriles, coches, meta, botón de inicio y podio.
    \item \textbf{Controlador:} \texttt{HelloController} gestiona la interacción entre la vista y la lógica, mueve los coches y actualiza el podio.
\end{itemize}

\subsection{Clases principales}
\begin{itemize}
    \item \texttt{HelloController}: arranca la carrera, actualiza posiciones de los coches en la interfaz y muestra el podio.
    \item \texttt{Coches}: cada coche es un hilo que calcula su avance, actualiza la posición en la UI y notifica al podio al terminar.
    \item \texttt{Grand\_Prix}: mantiene la lista de llegada de los coches y actualiza el podio sincronizado para evitar conflictos entre hilos.
\end{itemize}

\section{Synchronize}

Usamos \texttt{synchronized} para controlar el acceso al podio compartido:

\begin{verbatim}
// Clase Grand_Prix
public synchronized void ver_podio(String nombre) {
    Podio.add(nombre);
    if (Podio.size() == 4) {
        // Actualizar podio en UI
        Platform.runLater(() -> podioLabel.setText(sb.toString()));
        Podio.clear();
    }
}
\end{verbatim}

Esto asegura que cuando varios hilos intenten registrar llegada al mismo tiempo, no se produzcan errores de concurrencia.

\section{Interfaz}

El diseño se organiza en un \texttt{AnchorPane} con:

\begin{itemize}
    \item Cuatro carriles (\texttt{Rectangle})
    \item Meta (\texttt{Rectangle rojo})
    \item Etiquetas para los coches (\texttt{Label})
    \item Botón de inicio de carrera (\texttt{Button})
    \item Label para mostrar el podio (\texttt{Label})
\end{itemize}

Cada coche se mueve a lo largo del carril y la posición se actualiza en tiempo real usando \texttt{Platform.runLater}.

\section{Recursos gráficos}

Icono de coches y meta, se usan emojis integrados en Labels para representar los vehículos. 

\end{document}
